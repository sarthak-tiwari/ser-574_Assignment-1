%
% The first command in your LaTeX source must be the \documentclass command.
\documentclass[sigplan,screen]{acmart}

%
% defining the \BibTeX command - from Oren Patashnik's original BibTeX documentation.
\def\BibTeX{{\rm B\kern-.05em{\sc i\kern-.025em b}\kern-.08emT\kern-.1667em\lower.7ex\hbox{E}\kern-.125emX}}
    
% Rights management information. 
% This information is sent to you when you complete the rights form.
% These commands have SAMPLE values in them; it is your responsibility as an author to replace
% the commands and values with those provided to you when you complete the rights form.
%
% These commands are for a PROCEEDINGS abstract or paper.
\copyrightyear{2018}
\acmYear{2018}
\setcopyright{acmlicensed}
\acmConference[Woodstock '18]{Woodstock '18: ACM Symposium on Neural Gaze Detection}{June 03--05, 2018}{Woodstock, NY}
\acmBooktitle{Woodstock '18: ACM Symposium on Neural Gaze Detection, June 03--05, 2018, Woodstock, NY}
\acmPrice{15.00}
\acmDOI{10.1145/1122445.1122456}
\acmISBN{978-1-4503-9999-9/18/06}

%
% These commands are for a JOURNAL article.
%\setcopyright{acmcopyright}
%\acmJournal{TOG}
%\acmYear{2018}\acmVolume{37}\acmNumber{4}\acmArticle{111}\acmMonth{8}
%\acmDOI{10.1145/1122445.1122456}

%
% Submission ID. 
% Use this when submitting an article to a sponsored event. You'll receive a unique submission ID from the organizers
% of the event, and this ID should be used as the parameter to this command.
%\acmSubmissionID{123-A56-BU3}

%
% The majority of ACM publications use numbered citations and references. If you are preparing content for an event
% sponsored by ACM SIGGRAPH, you must use the "author year" style of citations and references. Uncommenting
% the next command will enable that style.
%\citestyle{acmauthoryear}

%
% end of the preamble, start of the body of the document source.
\begin{document}

%
% The "title" command has an optional parameter, allowing the author to define a "short title" to be used in page headers.
\title{Draft of Paper for SER - 574}

%
% The "author" command and its associated commands are used to define the authors and their affiliations.
% Of note is the shared affiliation of the first two authors, and the "authornote" and "authornotemark" commands
% used to denote shared contribution to the research.

\author{Ruben Acuna}
\affiliation{%
  \institution{Arizona State University}
  \streetaddress{Tempe}
  \city{Phoenix}
  \country{USA}}
\email{ruben.acuna@asu.edu}

\author{Sarthak Tiwari}
\affiliation{%
  \institution{Arizona State University}
  \streetaddress{Tempe}
  \city{Phoenix}
  \country{USA}}
\email{sarthak.tiwari@asu.edu}

%
% By default, the full list of authors will be used in the page headers. Often, this list is too long, and will overlap
% other information printed in the page headers. This command allows the author to define a more concise list
% of authors' names for this purpose.
\renewcommand{\shortauthors}{Ruben and Sarthak, et al.}

%
% The abstract is a short summary of the work to be presented in the article.
\begin{abstract}

With the boom in the use of agile process model, imperfect implementations of agile are frequently being seen, out of which the problem of inter-team communication in agile teams is one of the most common issue.
An agile team by its definition is a group of people who are self-sufficient to bring their responsibilities to closure, the interaction between different teams is thus considered minimal in most projects implementing agile.
This causes problems when the integration of end-products of different teams is carried out.
In this paper we have taken a detailed look at this problem and how we can mitigate this.

\end{abstract}

%
% The code below is generated by the tool at http://dl.acm.org/ccs.cfm.
% Please copy and paste the code instead of the example below.
%
\begin{CCSXML}
<ccs2012>
 <concept>
  <concept_id>10010520.10010553.10010562</concept_id>
  <concept_desc>Computer systems organization~Embedded systems</concept_desc>
  <concept_significance>500</concept_significance>
 </concept>
 <concept>
  <concept_id>10010520.10010575.10010755</concept_id>
  <concept_desc>Computer systems organization~Redundancy</concept_desc>
  <concept_significance>300</concept_significance>
 </concept>
 <concept>
  <concept_id>10010520.10010553.10010554</concept_id>
  <concept_desc>Computer systems organization~Robotics</concept_desc>
  <concept_significance>100</concept_significance>
 </concept>
 <concept>
  <concept_id>10003033.10003083.10003095</concept_id>
  <concept_desc>Networks~Network reliability</concept_desc>
  <concept_significance>100</concept_significance>
 </concept>
</ccs2012>
\end{CCSXML}

\ccsdesc[500]{Computer systems organization~Embedded systems}
\ccsdesc[300]{Computer systems organization~Redundancy}
\ccsdesc{Computer systems organization~Robotics}
\ccsdesc[100]{Networks~Network reliability}

%
% Keywords. The author(s) should pick words that accurately describe the work being
% presented. Separate the keywords with commas.
\keywords{Agile, Software Engineering, Process Models, Inter-team, Management}

%
% This command processes the author and affiliation and title information and builds
% the first part of the formatted document.
\maketitle

\section{Introduction}
%Introduction (Introducing the topic)

%introduce agile
Recently, agile approaches have become popular in software development.
As outlined in the Agile Manifesto \cite{beck2001agile}, an agile process gives individuals and interactions value over processes and tools.
%An agile team is supposed to contain all that is required for them to do their work, thus interactions no matter how valuable usually are done intra-team only.
As software grows in complexity, situations arise where multiple agile teams need to work together to achieve a common goal.
%explain what we are doing
In this paper we survey several the issues that can arise when agile teams must work together.
We also discuss approaches, which if implemented correctly, provide a more robust and functional process for inter-team agile development. 
%
%introduce the need to scale to multiple agile teams
%However, the stucture of agile team does not lend itself to large teams, and so applying the agile process requires mutiple teams.
Although an independent team may use an agile process with ease, additional considerations are required for complex systems where running multiple agile teams in parallel becomes necessary.
The agile approach of interacting in-person fails due to a few reasons such as the vertical structure of the company organization and the existence of middle managers \cite{dzone_article}.

%introduce potential issues (com, arch)
The transition to multiple teams increases the importance of inter-team communication, from simple interpersonal communication, to system specification. 
%arch
In an agile process, where Big Design Up-front may typically be avoided, the move to multiple teams motivates a stronger need to communicate design decisions.
In order for sucessful integration of techonological components, system elements must have well defined functional specifications, and interfaces.
%com
This large number of intermediaries in inter-team communication causes the process to fall apart as it takes longer than usual time for messages to reach their destination then what agile can afford.
This problem is difficult to fix as this is so embedded in the working agile culture of the current organizations that changing them will take a long time, thus an immediate patch that works is required.




%outline of rest of paper
This document is separated into five sections: Introduction, Process Inprovements, Technology Improvements, Importance of Architecture \& Design, and Conclusion. 
In Section ~\ref{sec:proc_impv}, we discuss the adaptations to an agile process, and team structure, to support better collaboration between teams.
In Section ~\ref{sec:tech_impv}, we discuss the ways in which technology improve inter-team communication.
In Section ~\ref{sec:imp_of_dsgn}, we discuss the role of architecture and design in mitigating communication issues. 

\section{Communication Concerns}\label{sec:spt_ex} 
%overview
A fundamental issue with tasking teams with different goals is that teams can become isolated with respect to the context of their work.
% problem
Consider a scenario at Spotify \cite{kniberg12}: a tester on a particular agile team invested time to solve a particular problem, which was also faced by another tester on a different team. Without communication, these team members would perform redundant work.
To address this problem Spotify defines Chapters, which are crosscutting groups which gather individuals with similar roles across different Squads. 
At Spotify \cite{kniberg12}, offices are laid out so that the Squads making up a Tribe are spatially close. This leads to an environment where an informal exchange of information on the Tribe's work will occur.
%Although larger than Squads (7-10 members), Tribes are limited in size (< 100) to ensure smooth communication.
%todo: any more issues? Dunbar

In general, consistency becomes an issue when expanding to multiple teams.
The product owner in an agile process model is the person with the project vision, and with a small team, the product owner can clearly communicate this to every team member.
Members can even query the product owner directly when in doubt.
However, in large projects where multiple teams are working together, it is practically impossible for the owner to continue what they did in a small team.
This motivates a need for a formal definition of their vision as it gives teams something to consult to when encountering a design decision.

\section{Supporting Communication in Inter-team Development}
\label{sec:prop_appro} 
The general problem of communication can be tackled from two perspectives.
First, stakeholders at and across different levels need to have communication channels.
This can be addressed by adaptations in development process and leveraging technology.
Second, the quality (e.g., accuracy) of communication needs to be high.
This can be addressed by thorough system specification through architecture.

\subsection{Process Adaptations} 
\label{sec:proc_impv}
	There are many changes \cite{collabAcrossAgile_article} that can be incorporated into a process model to decrease the chance that parallel agile teams run into problems during integration.

%Changes in standups
One critical aspect of an agile process is daily standups which are a platform for letting each team member know the work being done in other parts of the team and correlate it with the work they are performing.
This uncovers differences in understanding early in development and prevents last minute discovery of discrepancies.
In the case of multiple agile teams this problem is compounded as a stand up is a closed activity of the team itself, thus preventing other teams from knowing the results or discussions of each other.
This can be resolved by having a representative of each concerned team be present in the daily standup thus letting each team know the status of the other teams.

%Changes to Product Owner
Although the product owner is responsible for agile teams under his supervision, projects with multiple agile teams may have multiple product owners, and so over time the vision of the owners may drift and lead the development in different directions.
This can be limited by having regular meetings of product owners where the scope and vision of the project can be synchronized.
%This can be a bi-weekly or monthly meeting depending on the size of the project.

%Changes in Planning Sessions
All the initial, intermediate and final planning sessions should be organized such that teams which are, or could be, impacted by that part of the project are part of the meeting.
This can assist in early agreement on high-level requirements and standardization of inter-team interfaces.

\subsection{Technology Improvements} \label{sec:tech_impv}
Selecting an appropriate technology stack plays an important role in maintaining a distributed development environment.
Technology enables communication and task management so that teams can work together efficiently.
%The following are areas where technology can enable a more efficient agile process.

%Communication Tools
Since agile focuses on personal interaction, particularly face-to-face conversation, video conferencing tools can lower the bar to inter-team interactions.
They also make communication real-time, reducing delay due to time spent communicating ideas across teams.
%Integration Tools
Continuous integration tools can go a long way in identifying inter-team problems early in development
as it reveals the impact of changes on the entire project.


\subsection{Importance of Architecture and Design}\label{sec:imp_of_dsgn}
%Why is Design/Architecture so important when doing this?

System design and architecture acts as a common vision for not only teams but also for the communication between all the product owners.
The system design acts as a “deadlock breaker in decisions” \cite{architecureRole_article} for situations where teams disagree on a particular design path.
%non specification fix
A second approach addresses this issue from an organizational standpoint by defining a "system owner" role \cite{kniberg12}.
This role is more casual than an architect role, and focuses on defining a "go-to" person who can maintain a long term stewardship over a sub-system. 

In addition to specification, proper architecture can also be used to enable team agility.
As discussed by Parnas \cite{Parnas72}, there are two general approaches to decomposing systems: 1) compartmentalizing a computational process and, 2) focusing on information hiding.
The latter approach is ideal for agile development since it defines system components in terms of hiding design decisions, which empowers individual teams to design their own solution.

\section{Conclusion}\label{sec:conclusion}
As we have seen, inter-team communication can be an obstacle in agile teams working together, as intermediaries in communication and lack of personal communication causes a distruption in software process.
To reduce this problem a number of small additions can be done to the process as specified but the most important part of the change is a good architecture and design which is available to all the teams and thus provides a common framework to all the teams involved.

\section{CCS Concepts and User-Defined Keywords}

Two elements of the ``acmart'' document class provide powerful taxonomic tools for you to help readers find your work in an online search. 

The ACM Computing Classification System --- \url{https://www.acm.org/publications/class-2012} --- is a set of classifiers and concepts that describe the computing discipline. Authors can select entries from this classification system, via \url{https://dl.acm.org/ccs/ccs.cfm}, and generate the commands to be included in the \LaTeX\ source. 

User-defined keywords are a comma-separated list of words and phrases of the authors' choosing, providing a more flexible way of describing the research being presented.

CCS concepts and user-defined keywords are required for all short- and full-length articles, and optional for two-page abstracts. 

%
% The next two lines define the bibliography style to be used, and the bibliography file.
\bibliographystyle{ACM-Reference-Format}
\bibliography{sample-base}

\end{document}
