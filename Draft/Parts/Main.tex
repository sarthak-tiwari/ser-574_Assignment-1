\section{Communication Concerns}\label{sec:spt_ex} 
%overview
A fundamental issue in tasking teams with separate goals is that teams can become isolated from other teams pursuing similar or related goals.
% problem
Consider a scenario at Spotify \cite{kniberg12}: a tester in a particular Squad invested time to solve a particular problem, which was also faced by another tester on a different team. Without communication, these team members would perform redundant work.
To address this problem Spotify defines Chapters, which are crosscutting groups which gather individuals with similar roles across different Squads. 
At Spotify, offices are laid out so that the Squads making up a Tribe are spatially close \cite{kniberg12}.
This leads to an environment where an informal exchange of information on the Tribe's work will occur.
%Although larger than Squads (7-10 members), Tribes are limited in size (< 100) to ensure smooth communication.
%todo: any more issues? Dunbar

Consistency becomes an issue when expanding to multiple teams.
The product owner in an agile process model is the person with the project vision, and with a small team, they can clearly communicate it to every team member.
Members can even query the product owner directly when in doubt.
However, in large projects where multiple teams are working together, it is practically impossible for the owner to continue what they did in a small team.
This motivates a need for a formal definition of their vision as it gives teams something to consult to when encountering a design decision.
Without a functional specification, teams will be unable to produce components which can be easily integrated.

\section{Supporting Communication in Inter-team Development}
\label{sec:prop_appro} 
The general problem of communication can be tackled from two perspectives.
First, stakeholders at and across different levels need to have communication channels.
This can be addressed by adaptations in development process and leveraging technology.
Second, the quality (e.g., accuracy) of communication needs to be high.
This can be addressed by thorough system specification through architecture.

\subsection{Process Adaptations} 
\label{sec:proc_impv}
	There are many changes \cite{collabAcrossAgile_article} that can be incorporated into a process model to decrease the chance that parallel agile teams run into problems during development or integration.

%Changes in standups
One critical aspect of an agile process is daily standups which are a platform for letting each team member know the work being done in other parts of the team and correlate it with the work they are performing.
This uncovers differences in understanding early in development and prevents last minute discovery of discrepancies.
In the case of multiple agile teams this problem is compounded as a stand up is a closed activity of the team itself, thus preventing other teams from knowing the results or discussions of each other.
This can be resolved by having a representative of each team be present in a standup, e.g., a Scrum of Scrums, where typical stand up questions are given but reframed for a team context (e.g., "What will my team do before we meet again.." \cite{Rubin12}).

%Changes to Product Owner
Although the product owner is responsible for agile teams under his supervision, projects with multiple agile teams may have multiple product owners, and so over time the vision of the owners may drift and lead the development in different directions.
This can be limited by having regular meetings of product owners where the scope and vision of the project can be synchronized.
%This can be a bi-weekly or monthly meeting depending on the size of the project.

%Changes in Planning Sessions
All the initial, intermediate and final planning sessions should be organized such that teams which are, or could be, impacted by that part of the project are part of the meeting.
This can assist in early agreement on high-level requirements and standardization of inter-team interfaces.

\subsection{Technology Improvements} \label{sec:tech_impv}
Selecting an appropriate technology stack plays an important role in enabling communication and task management so that teams can work together efficiently.
%The following are areas where technology can enable a more efficient agile process.

%Communication Tools
Since agile focuses on interactions, like face-to-face conversation, video conferencing tools can lower the bar to inter-team interactions.
They also make communication real-time, reducing delay due to time spent communicating ideas across teams.
%Integration Tools
Continuous integration tools can go a long way in identifying inter-team problems early in development
as it reveals the impact of changes on the entire project.


\subsection{Importance of Architecture and Design}\label{sec:imp_of_dsgn}
%Why is Design/Architecture so important when doing this?

System design or architecture acts as a common vision for not only the teams but also for the communication between all the product owners and thus is the theme or topic for most of the inter-team communication that happens in a multi-team agile environment.
Being such a frequent piece of discussion, it needs to be as formal and as clearly defined as possible as it is the system design that acts as a “deadlock breaker in decisions” \cite{architecureRole_article} for situations where teams disagree on a particular design path, and ensures that components will be constructed so they can be integrated into the whole.
%non specification fix
A second approach addresses this issue from an organizational standpoint by defining a "system owner" role \cite{kniberg12}.
This role is more casual than an architect role, and focuses on defining a "go-to" person who can maintain a long term stewardship over a sub-system. 

In addition to specification, proper architecture can also be used to enable team agility.
As discussed by Parnas \cite{Parnas72}, there are two general approaches to decomposing systems: 1) compartmentalizing a computational process and, 2) focusing on information hiding.
The latter approach is ideal for agile development since it defines system components in terms of hiding design decisions, which empowers individual teams to design their own solution.

Another way in which a formal architecture description helps in mitigating the communication problems is by minimizing the need for inter-team communication as a good design document is clear enough for a developer to incorporate the design in the system without the need of any other document.
Thus, a good design documents sometimes acts as a proxy for product owner when it comes to detailing the requirements of the customer.