\section{Process Improvements} 
\label{sec:proc_impv}
	There are many small changes \cite{collabAcrossAgile_article} that we can incorporate in our process model to make sure that parallel developing agile teams do not run into problems when they reach the integration phase.
	These changes are to be made part of the entire process and are not to be implemented only in the end.

\subsection{Changes in standups}
	One of the most critical aspect of agile process model is to have daily standups which are the platform serving the purpose of letting each team member know the work being done in the other parts of the team and corelate it with the work being done by them.
	This results in escalation of differences between the development early in the process and prevents end moment discovery of mismatches in interfaces and such.
	In the case of multiple agile teams this problem is compounded as usually a daily stand up is a closed activity of the team itself, thus preventing other teams from knowing the results or discussions of each other.
	This can be resolved by having a representative of each concerned team being present in the daily standup thus letting each team know the status of other teams.

\subsection{Changes to Product Owner}
	Though in usual implementations the product owner is responsible for agile teams under his supervision, in large projects with multiple agile teams where a number of product owners are present sometimes over time the vision of the owners may get too distinct from each other thus pushing the development track in different directions.
	This can be limited by having regular meetings of product owners where the scope and vision of the project could be synced again. This can be a bi-weekly or monthly meeting depending on the size of the project.

\subsection{Changes in Planning Sessions}
	All the initial, intermediate and final planning sessions should be made such that all the teams which are or could be impacted by that part of the project are part of the meeting This can assist in early agreement on high-level requirements and standardization of inter-team interfaces.

\section{Technology Improvements} 
\label{sec:tech_impv}
	A good technology stack can be a powerful tool in maintaining a widely distributed team.
	Good communication and management tools can help the teams in keeping track of things that they need to do so that other teams can work as intended.
	The following are some of the areas where the technology can assist the teams in making a more efficient agile process environment.

\subsection{Communication Tools} As agile focuses on personal interaction with highest importance given to face-to-face conversation, conferencing tools such as video conferencing and WebEx etc. can help the team interactions become more fluid and clearer.
They also make the communication real-time thus removing the lag in process due to time spent in communicating ideas across teams.

\subsection{Integration Tools} Continuous integration tools can go a long way in finding out inter-team problems early in the process as every time any team makes a change, its impact on the entire project can be seen.

\section{Importance of Architecture/Design}
\label{sec:imp_of_dsgn}
As we know that the product owner in an agile process model is the person with the vision of what the project will look like and if it is a small team the product owner can clearly pass this to each and every member of the team and even each member can query the product owner directly when in doubt.
But in large scale projects where multiple agile teams are working together in supervision of a few product owners it is practically impossible for the owner to keep doing what they did in a small team, that’s where a formal definition of their vision comes in handy as it enables the teams to look up to something when encountering a design decision.
The design or architecture in this case acts as a common vision for not only the teams but also for the communication between all the product owners.
The design acts as a “deadlock breaker in decisions” \cite{architecureRole_article} as when the teams can’t come to a common consensus the design shows the path to take.
