%Introduction (Introducing the topic)

%introduce agile
Recently, agile approaches have become popular in software development.
As outlined in the Agile Manifesto \cite{beck2001agile}, an agile process gives individuals and interactions value over processes and tools.
%An agile team is supposed to contain all that is required for them to do their work, thus interactions no matter how valuable usually are done intra-team only.
As software grows in complexity, situations arise where multiple agile teams need to work together to achieve a common goal.
%explain what we are doing
In this paper we survey several the issues that can arise when agile teams must work together.
We also discuss approaches, which if implemented correctly, provide a more robust and functional process for inter-team agile development. 
%
%introduce the need to scale to multiple agile teams
%However, the stucture of agile team does not lend itself to large teams, and so applying the agile process requires mutiple teams.
Although an independent team may use an agile process with ease, additional considerations are required for complex systems where running multiple agile teams in parallel becomes necessary.
The agile approach of interacting in-person fails due to a few reasons such as the vertical structure of the company organization and the existence of middle managers \cite{dzone_article}.

%introduce potential issues (com, arch)
The transition to multiple teams increases the importance of inter-team communication, from simple interpersonal communication, to system specification. 
%arch
In an agile process, where Big Design Up-front may typically be avoided, the move to multiple teams motivates a stronger need to communicate design decisions.
In order for sucessful integration of techonological components, system elements must have well defined functional specifications, and interfaces.
%com
This large number of intermediaries in inter-team communication causes the process to fall apart as it takes longer than usual time for messages to reach their destination then what agile can afford.
This problem is difficult to fix as this is so embedded in the working agile culture of the current organizations that changing them will take a long time, thus an immediate patch that works is required.




%outline of rest of paper
This document is separated into four sections: Introduction, Sportify: A Canonical Example, Proposed Approach and Conclusion. 
In Section ~\ref{sec:spt_ex}, we discuss the a real world example of large scale agile project and show a problem encountered in that.
In Section ~\ref{sec:prop_appro}, we present some adaptations that can be done in agile process model to resolve the problem discussed.