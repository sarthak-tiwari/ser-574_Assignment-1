As mentioned in the Agile Manifesto \cite{beck2001agile}, Individuals and Interactions are given more value than the processes and tools. At the same time, an agile team is supposed to contain all that is required for them to do their work, thus interactions no matter how valuable usually are done intra-team only. Thus, when a situation arrives where two or more agile teams need to work together to achieve a common goal the usual approach taken by companies fail. The agile approach of interacting in-person fails due to a few reasons such as the vertical structure of the company organization and the existence of middle managers \cite{dzone_article}. This large number of intermediaries in inter team communication causes the process to crumble and fall apart as it takes longer than usual time for messages to reach their destination then what agile can afford.
This problem is harder to fix as this is so embedded in the working culture of the current organizations that changing them will take a long time, thus an immediate patch that works is required. And that’s where the approaches listed in the article comes in. We have collected a couple of approaches which if implemented efficiently results in a more stable and functioning agile process model. Thus, in the following sections we will be explaining couple of approaches we came across.
\linebreak
In Section ~\ref{sec:proc_impv} we will be explaining the improvements that can be made to the process and team structures to get better collaboration out of the team involved.
\linebreak
In section 3 we will be explaining the enhancements that can be done in the technology stack being used and in the different ways in which technology can help in improving the process.

